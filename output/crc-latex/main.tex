%********************************************%
%*       Generated from PreTeXt source      *%
%*       on 2025-02-18T14:14:42+01:00       *%
%*   A recent stable commit (2022-07-01):   *%
%* 6c761d3dba23af92cba35001c852aac04ae99a5f *%
%*                                          *%
%*         https://pretextbook.org          *%
%*                                          *%
%********************************************%
\documentclass[twoside,10pt,]{book}
%% Custom Preamble Entries, early (use latex.preamble.early)
%% Always open on odd page
%%   The following adjusts cleardoublepage to remove twosided
%%   check so that we open on odd pages even in one-sided mode
%%   by adding an extra blank page on the preceding even page.
\makeatletter%
\def\cleardoublepage{%
\clearpage\ifodd\c@page\else\thispagestyle{empty}\hbox{}\newpage\if@twocolumn\hbox{}\newpage\fi\fi%
}
\makeatother%
%% Default LaTeX packages
%%   1.  always employed (or nearly so) for some purpose, or
%%   2.  a stylewriter may assume their presence
\usepackage{geometry}
%% Some aspects of the preamble are conditional,
%% the LaTeX engine is one such determinant
\usepackage{ifthen}
%% etoolbox has a variety of modern conveniences
\usepackage{etoolbox}
\usepackage{ifxetex,ifluatex}
%% Raster graphics inclusion
\usepackage{graphicx}
%% Color support, xcolor package
%% Always loaded, for: add/delete text, author tools
%% Here, since tcolorbox loads tikz, and tikz loads xcolor
\PassOptionsToPackage{dvipsnames,svgnames,table}{xcolor}
\usepackage{xcolor}
%% begin: defined colors, via xcolor package, for styling
%% end: defined colors, via xcolor package, for styling
%% Colored boxes, and much more, though mostly styling
%% skins library provides "enhanced" skin, employing tikzpicture
%% boxes may be configured as "breakable" or "unbreakable"
%% "raster" controls grids of boxes, aka side-by-side
\usepackage{tcolorbox}
\tcbuselibrary{skins}
\tcbuselibrary{breakable}
\tcbuselibrary{raster}
%% We load some "stock" tcolorbox styles that we use a lot
%% Placement here is provisional, there will be some color work also
%% First, black on white, no border, transparent, but no assumption about titles
\tcbset{ bwminimalstyle/.style={size=minimal, boxrule=-0.3pt, frame empty,
colback=white, colbacktitle=white, coltitle=black, opacityfill=0.0} }
%% Second, bold title, run-in to text/paragraph/heading
%% Space afterwards will be controlled by environment,
%% independent of constructions of the tcb title
%% Places \blocktitlefont onto many block titles
\tcbset{ runintitlestyle/.style={fonttitle=\blocktitlefont\upshape\bfseries, attach title to upper} }
%% Spacing prior to each exercise, anywhere
\tcbset{ exercisespacingstyle/.style={before skip={1.5ex plus 0.5ex}} }
%% Spacing prior to each block
\tcbset{ blockspacingstyle/.style={before skip={2.0ex plus 0.5ex}} }
%% xparse allows the construction of more robust commands,
%% this is a necessity for isolating styling and behavior
%% The tcolorbox library of the same name loads the base library
\tcbuselibrary{xparse}
%% The tcolorbox library loads TikZ, its calc package is generally useful,
%% and is necessary for some smaller documents that use partial tcolor boxes
%% See:  https://github.com/PreTeXtBook/pretext/issues/1624
\usetikzlibrary{calc}
%% We use some more exotic tcolorbox keys to restore indentation to parboxes
\tcbuselibrary{hooks}
%% Save default paragraph indentation and parskip for use later, when adjusting parboxes
\newlength{\normalparindent}
\newlength{\normalparskip}
\AtBeginDocument{\setlength{\normalparindent}{\parindent}}
\AtBeginDocument{\setlength{\normalparskip}{\parskip}}
\newcommand{\setparstyle}{\setlength{\parindent}{\normalparindent}\setlength{\parskip}{\normalparskip}}%% Hyperref should be here, but likes to be loaded late
%%
%% Inline math delimiters, \(, \), need to be robust
%% 2016-01-31:  latexrelease.sty  supersedes  fixltx2e.sty
%% If  latexrelease.sty  exists, bugfix is in kernel
%% If not, bugfix is in  fixltx2e.sty
%% See:  https://tug.org/TUGboat/tb36-3/tb114ltnews22.pdf
%% and read "Fewer fragile commands" in distribution's  latexchanges.pdf
\IfFileExists{latexrelease.sty}{}{\usepackage{fixltx2e}}
%% Text height identically 9 inches, text width varies on point size
%% See Bringhurst 2.1.1 on measure for recommendations
%% 75 characters per line (count spaces, punctuation) is target
%% which is the upper limit of Bringhurst's recommendations
\geometry{letterpaper,total={340pt,9.0in}}
%% Custom Page Layout Adjustments (use publisher page-geometry entry)
\geometry{paper=letterpaper, layoutsize={7in,10in}, layoutoffset={.75in,.5in},  width=33pc, height=51pc, inner=.75in, outer=.75in, top=0.75in, showcrop=true}
%% This LaTeX file may be compiled with pdflatex, xelatex, or lualatex executables
%% LuaTeX is not explicitly supported, but we do accept additions from knowledgeable users
%% The conditional below provides  pdflatex  specific configuration last
%% begin: engine-specific capabilities
\ifthenelse{\boolean{xetex} \or \boolean{luatex}}{%
%% begin: xelatex and lualatex-specific default configuration
\ifxetex\usepackage{xltxtra}\fi
%% realscripts is the only part of xltxtra relevant to lualatex 
\ifluatex\usepackage{realscripts}\fi
%% end:   xelatex and lualatex-specific default configuration
}{
%% begin: pdflatex-specific default configuration
%% We assume a PreTeXt XML source file may have Unicode characters
%% and so we ask LaTeX to parse a UTF-8 encoded file
%% This may work well for accented characters in Western language,
%% but not with Greek, Asian languages, etc.
%% When this is not good enough, switch to the  xelatex  engine
%% where Unicode is better supported (encouraged, even)
\usepackage[utf8]{inputenc}
%% end: pdflatex-specific default configuration
}
%% end:   engine-specific capabilities
%%
%% Fonts.  Conditional on LaTex engine employed.
%% Default Text Font: The Latin Modern fonts are
%% "enhanced versions of the [original TeX] Computer Modern fonts."
%% We use them as the default text font for PreTeXt output.
%% Automatic Font Control
%% Portions of a document, are, or may, be affected by defined commands
%% These are perhaps more flexible when using  xelatex  rather than  pdflatex
%% The following definitions are meant to be re-defined in a style, using \renewcommand
%% They are scoped when employed (in a TeX group), and so should not be defined with an argument
\newcommand{\divisionfont}{\relax}
\newcommand{\blocktitlefont}{\relax}
\newcommand{\contentsfont}{\relax}
\newcommand{\pagefont}{\relax}
\newcommand{\tabularfont}{\relax}
\newcommand{\xreffont}{\relax}
\newcommand{\titlepagefont}{\relax}
%%
\ifthenelse{\boolean{xetex} \or \boolean{luatex}}{%
%% begin: font setup and configuration for use with xelatex
%% Generally, xelatex is necessary for non-Western fonts
%% fontspec package provides extensive control of system fonts,
%% meaning *.otf (OpenType), and apparently *.ttf (TrueType)
%% that live *outside* your TeX/MF tree, and are controlled by your *system*
%% (it is possible that a TeX distribution will place fonts in a system location)
%%
%% The fontspec package is the best vehicle for using different fonts in  xelatex
%% So we load it always, no matter what a publisher or style might want
%%
\usepackage{fontspec}
%%
%% begin: xelatex main font ("font-xelatex-main" template)
%% Latin Modern Roman is the default font for xelatex and so is loaded with a TU encoding
%% *in the format* so we can't touch it, only perhaps adjust it later
%% in one of two ways (then known by NFSS names such as "lmr")
%% (1) via NFSS with font family names such as "lmr" and "lmss"
%% (2) via fontspec with commands like \setmainfont{Latin Modern Roman}
%% The latter requires the font to be known at the system-level by its font name,
%% but will give access to OTF font features through optional arguments
%% https://tex.stackexchange.com/questions/470008/
%% where-and-how-does-fontspec-sty-specify-the-default-font-latin-modern-roman
%% http://tex.stackexchange.com/questions/115321
%% /how-to-optimize-latin-modern-font-with-xelatex
%%
%% end:   xelatex main font ("font-xelatex-main" template)
%% begin: xelatex mono font ("font-xelatex-mono" template)
%% (conditional on non-trivial uses being present in source)
%% end:   xelatex mono font ("font-xelatex-mono" template)
%% begin: xelatex font adjustments ("font-xelatex-style" template)
%% end:   xelatex font adjustments ("font-xelatex-style" template)
%%
%% Extensive support for other languages
\usepackage{polyglossia}
%% Set main/default language based on pretext/@xml:lang value
%% Enable secondary languages based on discovery of @xml:lang values
%% Enable fonts/scripts based on discovery of @xml:lang values
%% Western languages should be ably covered by Latin Modern Roman
%% end:   font setup and configuration for use with xelatex
}{%
%% begin: font setup and configuration for use with pdflatex
%% begin: pdflatex main font ("font-pdflatex-main" template)
\usepackage{lmodern}
\usepackage[T1]{fontenc}
%% end:   pdflatex main font ("font-pdflatex-main" template)
%% begin: pdflatex mono font ("font-pdflatex-mono" template)
%% (conditional on non-trivial uses being present in source)
%% end:   pdflatex mono font ("font-pdflatex-mono" template)
%% begin: pdflatex font adjustments ("font-pdflatex-style" template)
%% end:   pdflatex font adjustments ("font-pdflatex-style" template)
%% end:   font setup and configuration for use with pdflatex
}
%% Micromanage spacing, etc.  The named "microtype-options"
%% template may be employed to fine-tune package behavior
\usepackage{microtype}
%% Symbols, align environment, commutative diagrams, bracket-matrix
\usepackage{amsmath}
\usepackage{amscd}
\usepackage{amssymb}
%% allow page breaks within display mathematics anywhere
%% level 4 is maximally permissive
%% this is exactly the opposite of AMSmath package philosophy
%% there are per-display, and per-equation options to control this
%% split, aligned, gathered, and alignedat are not affected
\allowdisplaybreaks[4]
%% allow more columns to a matrix
%% can make this even bigger by overriding with  latex.preamble.late  processing option
\setcounter{MaxMatrixCols}{30}
%%
%%
%% Division Titles, and Page Headers/Footers
%% titlesec package, loading "titleps" package cooperatively
%% See code comments about the necessity and purpose of "explicit" option.
%% The "newparttoc" option causes a consistent entry for parts in the ToC 
%% file, but it is only effective if there is a \titleformat for \part.
%% "pagestyles" loads the  titleps  package cooperatively.
\usepackage[explicit, newparttoc, pagestyles]{titlesec}
%% The companion titletoc package for the ToC.
\usepackage{titletoc}
%% Fixes a bug with transition from chapters to appendices in a "book"
%% See generating XSL code for more details about necessity
\newtitlemark{\chaptertitlename}
%% begin: customizations of page styles via the modal "titleps-style" template
%% Designed to use commands from the LaTeX "titleps" package
%% Page style configuration
%%
%% Plain pages should have the same font for page numbers
\renewpagestyle{plain}{%
\setfoot[\pagefont\thepage][][]{}{}{\pagefont\thepage}%
}%
\renewpagestyle{headings}{\sethead[\pagefont\thepage][][\itshape\chaptertitle]{\itshape\sectiontitle}{}{\pagefont\thepage}}\pagestyle{headings}
%% end: customizations of page styles via the modal "titleps-style" template
%%
%% Create globally-available macros to be provided for style writers
%% These are redefined for each occurence of each division
\newcommand{\divisionnameptx}{\relax}%
\newcommand{\titleptx}{\relax}%
\newcommand{\subtitleptx}{\relax}%
\newcommand{\shortitleptx}{\relax}%
\newcommand{\authorsptx}{\relax}%
\newcommand{\epigraphptx}{\relax}%
%% Create environments for possible occurences of each division
%% Environment for a PTX "chapter" at the level of a LaTeX "chapter"
\NewDocumentEnvironment{chapterptx}{mmmmmmm}
{%
\renewcommand{\divisionnameptx}{#1}%
\renewcommand{\titleptx}{#2}%
\renewcommand{\subtitleptx}{#3}%
\renewcommand{\shortitleptx}{#4}%
\renewcommand{\authorsptx}{#5}%
\renewcommand{\epigraphptx}{#6}%
\chapter[{#4}]{#2}%
\label{#7}%
}{}%
%%
%% Styles for six traditional LaTeX divisions
\titleformat{\part}[display]
{\divisionfont\Huge\bfseries\centering}{\divisionnameptx\space\thepart}{30pt}{\Huge#1}
[{\Large\centering\authorsptx}]
\titleformat{\chapter}[display]
    {\Huge}
    {\bfseries\thechapter} 
    {-12pt}
    {\hbox to\textwidth{\rlap{\rule[-3.5pt]{84pt}{4pt}}\rule{\textwidth}{.5pt}}
     \vspace{0ex}\slshape #1}
    [\vspace{-1ex}\noindent\hbox{\vrule height.5pt width84pt}]
    \titlespacing*{\chapter}{0pt}{-2em}{5em}
\titleformat{name=\chapter,numberless}[display]
      {\huge}
      {} 
      {-12pt}
      {\hbox to\textwidth{\rlap{\rule[-3.5pt]{84pt}{4pt}}\rule{\textwidth}{.0pt}}\vspace{0ex}\slshape #1}
      [\vspace{-1ex}\noindent\hbox{\vrule height.5pt width84pt}]
      \titlespacing*{\chapter}{0pt}{-2em}{2em}
\titlespacing*{\chapter}{0pt}{-2em}{2em}
\titleformat{\subparagraph}[block]
      {\normalfont\filcenter\scshape\bfseries}
      {\theparagraph} 
      {1em}
      {#1}
      [\normalsize\authorsptx]
\titleformat{\section}[display]
    {\Large\bfseries}
    {}
    {-2em}
    {\hbox to\textwidth{\rlap{\rule[-3.5pt]{84pt}{4pt}}\rule{\textwidth}{.5pt}}
    \vspace{1ex}\thesection~ #1}
    [\large\authorsptx]
\titleformat{name=\section,numberless}
    {\filcenter\scshape\bfseries}
    {}
    {0.0em}
    {#1}
\titleformat{\subsection}
    {\large\bfseries}
    {\thesubsection}
    {1em}
    {#1}
    [\normalsize\authorsptx]
\titleformat{\subsubsection}
      {\large\bfseries}
      {\thesubsubsection}
      {1em}
      {#1}
      [\normalsize\authorsptx]
\titleformat{\paragraph}[hang]
{\divisionfont\normalsize\bfseries}{\theparagraph}{1em}{#1}
[{\small\authorsptx}]
\titleformat{name=\paragraph,numberless}[block]
{\divisionfont\normalsize\bfseries}{}{0pt}{#1}
[{\normalsize\authorsptx}]
\titlespacing*{\paragraph}{0pt}{3.25ex plus 1ex minus .2ex}{1.5em}
%%
%% Styles for five traditional LaTeX divisions
\titlecontents{part}%
[0pt]{\contentsmargin{0em}\addvspace{1pc}\contentsfont\bfseries}%
{\Large\thecontentslabel\enspace}{\Large}%
{}%
[\addvspace{.5pc}]%
\titlecontents{chapter}%
[0pt]{\contentsmargin{0em}\addvspace{1pc}\contentsfont\bfseries}%
{\large\thecontentslabel\enspace}{\large}%
{\hfill\bfseries\thecontentspage}%
[\addvspace{.5pc}]%
\dottedcontents{section}[3.8em]{\contentsfont}{2.3em}{1pc}%
\dottedcontents{subsection}[6.1em]{\contentsfont}{3.2em}{1pc}%
\dottedcontents{subsubsection}[9.3em]{\contentsfont}{4.3em}{1pc}%
%%
%% Begin: Semantic Macros
%% To preserve meaning in a LaTeX file
%%
%% \mono macro for content of "c", "cd", "tag", etc elements
%% Also used automatically in other constructions
%% Simply an alias for \texttt
%% Always defined, even if there is no need, or if a specific tt font is not loaded
\newcommand{\mono}[1]{\texttt{#1}}
%%
%% Following semantic macros are only defined here if their
%% use is required only in this specific document
%%
%% Style of a title on a list item, for ordered and unordered lists
%% Also "task" of exercise, PROJECT-LIKE, EXAMPLE-LIKE
\newcommand{\lititle}[1]{{\slshape#1}}
%% End: Semantic Macros
%% Publisher file requests an alternate chapter numbering
\setcounter{chapter}{-1}
%% Equation Numbering
%% Controlled by  numbering.equations.level  processing parameter
%% No adjustment here implies document-wide numbering
\numberwithin{equation}{chapter}
%% More flexible list management, esp. for references
%% But also for specifying labels (i.e. custom order) on nested lists
\usepackage{enumitem}
%% hyperref driver does not need to be specified, it will be detected
%% Footnote marks in tcolorbox have broken linking under
%% hyperref, so it is necessary to turn off all linking
%% It *must* be given as a package option, not with \hypersetup
\usepackage[hyperfootnotes=false]{hyperref}
%% For a print PDF, no surrounding boxes, so simply textcolor (but still active to preserve spacing)
\hypersetup{hidelinks}
%% Less-clever names for hyperlinks are more reliable, *especially* for structural parts
%% See comments in the code to learn more about the importance of this setting
\hypersetup{hypertexnames=false}
%%The  hypertexnames  setting then confuses the hyperlinking from the index
%%This patch resolves the incorrect links, see code for StackExchange post.
\makeatletter
\patchcmd\Hy@EveryPageBoxHook{\Hy@EveryPageAnchor}{\Hy@hypertexnamestrue\Hy@EveryPageAnchor}{}{\fail}
\makeatother
\hypersetup{pdftitle={Probabilités discrètes}}
%% If you manually remove hyperref, leave in this next command
%% This will allow LaTeX compilation, employing this no-op command
\providecommand\phantomsection{}
%% Division Numbering: Chapters, Sections, Subsections, etc
%% Division numbers may be turned off at some level ("depth")
%% A section *always* has depth 1, contrary to us counting from the document root
%% The latex default is 3.  If a larger number is present here, then
%% removing this command may make some cross-references ambiguous
%% The precursor variable $numbering-maxlevel is checked for consistency in the common XSL file
\setcounter{secnumdepth}{2}
%%
%% AMS "proof" environment is no longer used, but we leave previously
%% implemented \qedhere in place, should the LaTeX be recycled
\newcommand{\qedhere}{\relax}
%%
%% A faux tcolorbox whose only purpose is to provide common numbering
%% facilities for most blocks (possibly not projects, 2D displays)
%% Controlled by  numbering.theorems.level  processing parameter
\newtcolorbox[auto counter, number within=section]{block}{}
%%
%% This document is set to number PROJECT-LIKE on a separate numbering scheme
%% So, a faux tcolorbox whose only purpose is to provide this numbering
%% Controlled by  numbering.projects.level  processing parameter
\newtcolorbox[auto counter, number within=chapter]{project-distinct}{}
%% A faux tcolorbox whose only purpose is to provide common numbering
%% facilities for 2D displays which are subnumbered as part of a "sidebyside"
\makeatletter
\newtcolorbox[auto counter, number within=tcb@cnt@block, number freestyle={\noexpand\thetcb@cnt@block(\noexpand\alph{\tcbcounter})}]{subdisplay}{}
\makeatother
%%
%% tcolorbox, with styles, for inline exercises
%%
%% inlineexercise: fairly simple numbered block/structure
\tcbset{ inlineexercisestyle/.style={
      enhanced,
      breakable,
      parbox=false,
      frame hidden,
      borderline west={1.5pt}{0mm}{black!90},
      overlay unbroken and last={
        \draw[black, path fading=east, line width=1pt] (frame.south west) -- (frame.south east);
      },
      colback=white,
      coltitle=white,
      fonttitle=\bfseries\sffamily,
      attach boxed title to top left={xshift=0mm},
      boxed title style={colback=black!90, sharp corners, colframe=black!90},
      boxed title size=title,
      after skip=1em,
      before skip=1em,
    } }
\newtcolorbox[use counter from=block]{inlineexercise}[3]{title={{#1~\thetcbcounter\notblank{#2}{\space\space#2}{}}}, phantomlabel={#3}, breakable, after={\par}, inlineexercisestyle, }
%% Graphics Preamble Entries
\usepackage{adjustbox}
\usepackage{tikz, pgfplots}
\pgfplotsset{compat=1.18}
\usetikzlibrary{positioning,matrix,arrows}
\usetikzlibrary{shapes,decorations,shadows,fadings,patterns}
\usetikzlibrary{decorations.markings}
%% If tikz has been loaded, replace ampersand with \amp macro
%% Custom Preamble Entries, late (use latex.preamble.late)
%This should load all the style information that ptx does not.
\input{external/latex-preamble-styles-crc}

%% extpfeil package for certain extensible arrows,
%% as also provided by MathJax extension of the same name
%% NB: this package loads mtools, which loads calc, which redefines
%%     \setlength, so it can be removed if it seems to be in the 
%%     way and your math does not use:
%%     
%%     \xtwoheadrightarrow, \xtwoheadleftarrow, \xmapsto, \xlongequal, \xtofrom
%%     
%%     we have had to be extra careful with variable thickness
%%     lines in tables, and so also load this package late
\usepackage{extpfeil}
%% Begin: Author-provided TeX/LaTeX packages
%% (From  docinfo/math-package  elements)
\usepackage{mathrsfs}
\usepackage{mathtools}
\usepackage{empheq}
%% End: Author-provided TeX/LaTeX packages
%% Begin: Author-provided macros
%% (From  docinfo/macros  element)
%% Plus three from PTX for XML characters
\newcommand{\N}{\mathbb N}
\newcommand{\Z}{\mathbb Z}
\newcommand{\Q}{\mathbb Q}
\newcommand{\R}{\mathbb R}
\newcommand{\C}{\mathbb C}
\newcommand{\K}{\mathbb K}
\renewcommand{\Pr}{\mathbb P}
\newcommand{\Es}{\mathbb E}
\newcommand{\Va}{\mathbb V}
\newcommand{\PP}{\mathbb P}
\newcommand{\ES}{\mathbb E}
\newcommand{\VA}{\mathbb V}
\newcommand\Cov{\operatorname{Cov}}
\newcommand\card{\operatorname{Card}}
\newcommand\qtext[1]{\quad\text{#1}\quad}
\renewcommand\llbracket{[\![}
\renewcommand\rrbracket{]\!]}
\newcommand\iic[1]{\llbracket #1 \rrbracket}
\let\geq\geqslant
\let\leq\leqslant
\let\ds\displaystyle
\newcommand{\lt}{<}
\newcommand{\gt}{>}
\newcommand{\amp}{&}
%% End: Author-provided macros
\begin{document}
%% bottom alignment is explicit, since it normally depends on oneside, twoside
\raggedbottom
%
%
\typeout{************************************************}
\typeout{Chapitre 0 Exercices : Probabilités discrètes}
\typeout{************************************************}
%
\begin{chapterptx}{Chapitre}{Exercices : Probabilités discrètes}{}{Exercices : Probabilités discrètes}{}{}{ch-exercices}
\renewcommand*{\chaptername}{Chapitre}
\begin{inlineexercise}{Exercice}{Une caractérisation de la loi de Poisson.}{ch-exercices-2}%
On considère une variable aléatoire discrète \(N\) sur l'espace probabilisé \((\Omega, \mathcal{A}, \Pr)\) telle que \(N(\Omega)=\N\) et \(\Pr(N=n) \neq 0\) pour tout \(n \in \N\). Si la variable aléatoire \(N\) prend la valeur \(n\), on procède à une succession de \(n\) épreuves de Bernoulli indépendantes de paramètre \(p \in] 0,1[\). On note \(S\) et \(E\) les variables aléatoires représentant respectivement le nombre de succès et d'échecs dans ces \(n\) épreuves.%
\begin{enumerate}[font=\bfseries,label=(\alph*),ref=\alph*]%
\item{}Montrer que si \(N\) suit une loi de Poisson de paramètre \(\lambda>0\), les variables \(S\) et \(E\) suivent aussi des lois de Poisson dont on déterminera les paramètres. Montrer que les variables \(E\) et \(S\) sont indépendantes.%
\item{}Montrer réciproquement que si \(S\) et \(E\) sont indépendantes, alors \(N\) suit une loi de Poisson. Pour cela, on montrera :%
%
\begin{itemize}[label=\textbullet]
\item{}qu'il existe deux suites \(\left(u_{n}\right)_{n \in \N}\) et \(\left(v_{n}\right)_{n \in \N}\) telles que :%
\begin{equation*}
\forall(m, n) \in \N^{2} \quad(m+n) ! \Pr(N=m+n)=u_{m} v_{n}
\end{equation*}
%
\item{}que les suites \(\left(u_{n}\right)_{n \in \N}\) et \(\left(v_{n}\right)_{n \in \N}\) sont géométriques.%
\end{itemize}
\end{enumerate}%
\end{inlineexercise}%
\begin{inlineexercise}{Exercice}{Lancer de pièce jusqu'au premier pile.}{ch-exercices-3}%
On lance une pièce de monnaie jusqu'à l'obtention du premier pile, la probabilité d'obtenir pile étant \(p \in] 0,1[\). Soit \(N\) la variable aléatoire représentant le nombre de lancers nécessaires. Si \(N=n\), on relance ensuite \(n\) fois la pièce et on appelle \(X\) la variable aléatoire représentant le nombre de piles obtenu.%
Déterminer la loi de \(N\), celle du couple \((N, X)\), puis la loi de \(X\).%
Montrer que \(X\) a même loi que le produit de deux variables indépendantes \(Y\) et \(Z\) telles que \(Y\) suive une loi de Bernoulli et \(Z\) une loi géométrique de même paramètre.%
En déduire l'espérance et la variance de \(X\).%
\end{inlineexercise}%
\begin{inlineexercise}{Exercice}{Taux de panne.}{ch-exercices-4}%
Soit \(X\) une variable aléatoire discrète à valeurs dans \(\N^{*}\) vérifiant :%
\begin{equation*}
\forall n \in \N^{*} \quad \Pr(X \geqslant n)>0
\end{equation*}
\(X\) représente le moment où un mécanisme tombe en panne. C'est à dire le numéro de l'instance de son cycle de fonctionnement où il tombe en panne. En principe, sous l'effet de l'usure, plus la durée de son fonctionnement est grande plus la probabilité que le mécanisme tombe en panne augmente.%
 \par
que On appelle taux de panne associé à \(X\) la suite réelle \(\left(x_{n}\right)_{n \in \N^{*}}\) définie par :%
\begin{equation*}
\forall n \in \N^{*} \quad x_{n}=\Pr(X=n \mid X \geqslant n)
\end{equation*}
\(x_n\) est la probabilité pour que le mécanisme tombe en panne à l'instant \(n\) sachant qu'il a fonctionné jusqu'à cet instant.%
\begin{enumerate}[font=\bfseries,label=(\alph*),ref=\alph*]%
\item{}Exprimer \(p_{n}=\Pr(X=n)\) en fonction des \(x_{k}\).%
\par\smallskip%
\noindent\textbf{\blocktitlefont Indication}.\hypertarget{ch-exercices-4-3-2}{}\quad{}Éviter de diviser par \(x_n\). Exprimer \(\Pr(X\geq n)\) comme un produit de facteurs \((1-x_k)\).%
\item{}\lititle{Caractérisation du taux de panne.}\par%
%
\begin{enumerate}[label={\arabic*.}]
\item{}Montrer que \(0 \leqslant x_{n} \lt 1\) pour tout \(n \in \N^{*}\) et que la série de terme général \(x_{n}\) diverge.%
\item{}Réciproquement, soit \(\left(x_{n}\right)_{n \in \N^{*}}\) une suite à valeur dans \([0,1[\) telle que la série de terme général \(x_{n}\) diverge. Montrer qu'il existe une variable aléatoire dont le taux de panne est la suite \(\left(x_{n}\right)\).%
\end{enumerate}
%
\par\smallskip%
\noindent\textbf{\blocktitlefont Indication}.\hypertarget{ch-exercices-4-4-3}{}\quad{}On rappelle que pour une suite \((p_n)_n\) de réels positifs sommable et de somme \(1\), il existe une variable aléatoire \(X\) telle que \(\Pr(X=n)=p_n\) pour tout \(n\).%
\item{}Montrer que la variable \(X\) suit une loi géométrique si, et seulement si, son taux de panne est constant.%
\end{enumerate}%
\end{inlineexercise}%
\begin{inlineexercise}{Exercice}{Maximums et minimums provisoires.}{ch-exercices-5}%
Soit \(n \in \N^{*}\). On désigne par \(\Omega\) l'ensemble des permutations de \(\llbracket 1, n \rrbracket\). On munit \(\Omega\) de la probabilité uniforme. Pour \(\sigma \in \Omega\) et \(i \in \llbracket 1, n \rrbracket\), on dit que \(\sigma(i)\) est un maximum (resp. minimum) provisoire de \(\sigma\) si :%
\begin{equation*}
\sigma(i) = \max (\sigma(1), \sigma(2), \ldots, \sigma(i))
\end{equation*}
(resp. \(\sigma(i) = \min (\sigma(1), \sigma(2), \ldots, \sigma(i))\)). On désigne par \(X_{n}\) (resp. \(Y_{n}\)) les variables aléatoires représentant le nombre de maximums (resp. minimums) provisoires des permutations de \(\llbracket 1, n \rrbracket\).%
Montrer que les variables \(X_{n}\) et \(Y_{n}\) ont même loi.%
Déterminer la loi de \(X_{3}\), son espérance et sa variance.%
Déterminer la loi du couple \((X_{3}, Y_{3})\) et sa covariance.%
Pour \(n \in \N^{*}\), on note \(g_{n}\) la fonction génératrice de \(X_{n}\).%
%
\begin{enumerate}[label={\arabic*.}]
\item{}Pour \(1 \leqslant k \leqslant n\), on note \(Z_{k}\) la variable indicatrice de l'événement « \(\sigma(k)\) est un maximum ». Montrer que les variables \(Z_{1}, Z_{2}, \ldots, Z_{n}\) sont indépendantes.%
\item{}Exprimer \(X_{n}\) en fonction de \(Z_{1}, Z_{2}, \ldots, Z_{n}\). En déduire \(g_{n}\).%
\item{}En déduire \(\Pr(X_{n}=1)\), \(\Pr(X_{n}=2)\), \(\Pr(X_{n}=n)\).%
\item{}Déterminer \(\Es(X_{n})\) et \(\Va(X_{n})\) (sous forme de sommes) et un équivalent de \(\Es(X_{n})\) et de \(\Va(X_{n})\) quand \(n\) tend vers \(+\infty\).%
\end{enumerate}
\end{inlineexercise}%
\begin{inlineexercise}{Exercice}{Formule du crible.}{ch-exercices-6}%
Soit \(A_{1}, A_{2}, \ldots, A_{n}\) des événements d'un espace probabilisé \((\Omega, \mathcal{A}, \Pr)\).%
Montrer que \(1_{\bigcup_{i=1}^{n} A_{i}} = 1 - \prod_{i=1}^{n} (1 - \mathbf{1}_{A_{i}})\). En déduire la formule du crible :%
\begin{equation*}
\Pr\Big(\bigcup_{i=1}^{n} A_{i}\Big) = \sum_{k=1}^{n} \bigg((-1)^{k-1} \sum_{I \subset [1, n] \atop \card I = k} \Pr\Big(\bigcap_{i \in I} A_{I}\Big)\bigg).
\end{equation*}
%
Soient \(n \in \N^{*}\) et \((X_{k})_{k \in \N^{*}}\) une suite de variables indépendantes d'un espace probabilisé \((\Omega, \mathcal{A}, \Pr)\) suivant toutes la loi uniforme sur \(\llbracket 1, n \rrbracket\). On note \(X\) la variable aléatoire égale au nombre de tirages nécessaires pour obtenir tous les numéros entre 1 et \(n\) au moins une fois (et à \(+\infty\) si on n'obtient jamais les \(n\) numéros). Pour \(j \in \llbracket 1, n \rrbracket\) et \(m \in \N\), on note \(B_{j, m}\) l'événement : « au bout de \(m\) tirages, le numéro \(j\) n'est pas encore apparu ».%
%
\begin{enumerate}[label={\arabic*.}]
\item{}Calculer \(\Pr(B_{j_{1}, m} \cap B_{j_{2}, m} \cap \cdots \cap B_{j_{k}, m})\) où \(j_{1}, j_{2}, \ldots, j_{k}\) sont des indices distincts compris entre 1 et \(n\).%
\item{}En déduire que :%
\begin{equation*}
\Pr(X \gt m) = \sum_{k=1}^{n} (-1)^{k-1} \binom{n}{k} \Big(\frac{n - k}{n}\Big)^{m}.
\end{equation*}
Calculer \(\lim_{m \rightarrow +\infty} \Pr(X \gt m)\). Interpréter.%
\item{}Montrer que \(\Es(X) = n \sum_{k=1}^{n} (-1)^{k-1} \frac{\binom{n}{k}}{k}\).%
\item{}Montrer que \(\Es(X) = n \left(1 + \frac{1}{2} + \cdots + \frac{1}{n}\right)\). En déduire un équivalent de \(\Es(X)\) quand \(n\) tend vers \(+\infty\).%
\end{enumerate}
\end{inlineexercise}%
\begin{inlineexercise}{Exercice}{Variables aléatoires uniformes et Poisson.}{ch-exercices-7}%
Soient un entier \(n \geqslant 1\) et une suite \((U_{k})_{k \in \N^{*}}\) de variables aléatoires indépendantes et de même loi uniforme sur \(\llbracket 1, n \rrbracket\). Pour tout \(i \in \llbracket 1, n \rrbracket\), on définit :%
\begin{equation*}
X_{i}^{(0)} = 0 \quad \text{et} \quad X_{i}^{(m)} = \card\{k \in \llbracket 1, m \rrbracket \mid U_{k} = i\} \quad \forall m \geqslant 1.
\end{equation*}
%
Quelle est la loi de \(X_{i}^{(m)}\) pour \(i \in \llbracket 1, n \rrbracket\) et \(m \geqslant 1\) ?%
Soit \(m \geqslant 1\) et \((i, j) \in \llbracket 1, n \rrbracket^{2}\) avec \(i \neq j\). Calculer la covariance des variables aléatoires \(X_{i}^{(m)}\) et \(X_{j}^{(m)}\). Sont-elles indépendantes ?%
Soit \(\lambda\gt0\) et \(N\) une variable aléatoire suivant une loi de Poisson de paramètre \(\lambda\), indépendante des variables \(U_{k}\). On pose :%
\begin{equation*}
\forall i \in \llbracket 1, n \rrbracket \quad Y_{i} = X_{i}^{(N)}.
\end{equation*}
%
%
\begin{enumerate}[label={\arabic*.}]
\item{}Déterminer, en fonction de \(\lambda\) et \(n\), la loi de \(Y_{i}\) pour tout \(i \in \llbracket 1, n \rrbracket\).%
\item{}Déterminer la loi conjointe de \((Y_{1}, \ldots, Y_{n})\).%
\end{enumerate}
\end{inlineexercise}%
\begin{inlineexercise}{Exercice}{Centrale 2015.}{ch-exercices-8}%
Soit \((\Omega, \mathcal{A}, \Pr)\) un espace probabilisé et \((E_{n})_{n \in \N} \in \mathcal{A}^{\N}\) une suite d'événements. On suppose que \(\sum_{n=0}^{+\infty} \Pr(E_{n}) \lt +\infty\), c'est-à-dire que la série converge.%
On note \(1_{X}\) la fonction indicatrice d'un ensemble \(X\). Soit \(Z = \sum_{n=0}^{+\infty} 1_{E_{n}}\) (on convient que \(Z = +\infty\) si la série diverge). Prouver que \(Z\) est une variable aléatoire discrète.%
Soit \(F = \{\omega \in \Omega \mid \omega \text{ appartient à un nombre fini de } E_{n}\}\). Prouver que \(F\) est un événement et que \(\Pr(F) = 1\).%
Prouver que \(Z\) est d'espérance finie.%
\end{inlineexercise}%
\begin{inlineexercise}{Exercice}{Marche aléatoire dans \(\mathbb{Z}\).}{ch-exercices-9}%
Soit \((X_{n})_{n \in \N^{*}}\) une suite de variables aléatoires, sur le même espace probabilisé \((\Omega, \mathcal{A}, \Pr)\), indépendantes et de même loi définie par :%
\begin{equation*}
\Pr(X_{n} = 1) = p \quad \text{et} \quad \Pr(X_{n} = -1) = 1 - p,
\end{equation*}
où \(p \in [0, 1]\). On pose \(S_{0} = 0\) et, pour tout \(n \in \N^{*}\), \(S_{n} = \sum_{k=1}^{n} X_{k}\). La suite \((S_{n})\) est appelée marche aléatoire dans \(\mathbb{Z}\).%
Déterminer \(u_{n} = \Pr(S_{n} = 0)\) pour tout \(n \in \N\).%
On note \(f(x)\) la somme de la série entière \(\sum u_{n} x^{n}\). Montrer que :%
\begin{equation*}
\forall x \in ]-1, 1[ \quad f(x) = \frac{1}{\sqrt{1 - 4 p (1 - p) x^{2}}}.
\end{equation*}
%
Pour tout entier naturel non nul \(k\), on note \(A_{k}\) l'événement « le mobile retourne pour la première fois à l'origine au bout de \(k\) déplacements », c'est-à-dire :%
\begin{equation*}
A_{k} = (S_{k} = 0) \cap \left(\bigcap_{i=1}^{k-1} (S_{i} \neq 0)\right).
\end{equation*}
On pose \(v_{k} = \Pr(A_{k})\) pour tout \(k \geqslant 1\) et \(v_{0} = 0\).%
%
\begin{enumerate}[label={\arabic*.}]
\item{}Montrer que, pour tout entier naturel \(n\) non nul, on a :%
\begin{equation*}
(S_{n} = 0) = \sum_{k=1}^{n} \Pr((S_{n} = 0) \cap A_{k}).
\end{equation*}
%
\item{}En déduire que, pour tout entier naturel non nul \(n\), on a :%
\begin{equation*}
u_{n} = \sum_{k=0}^{n} u_{n - k} v_{k}.
\end{equation*}
%
\end{enumerate}
\end{inlineexercise}%
\begin{inlineexercise}{Exercice}{Loi faible des grands nombres dans \(L_{1}\).}{ch-exercices-10}%
Soit \((X_{n})_{n \geqslant 1}\) une suite de variables aléatoires réelles discrètes, deux à deux indépendantes, de même loi, possédant une espérance finie \(m\). On pose, pour tout \(n \in \N^{*}\), \(Y_{n} = \frac{1}{n} (X_{1} + \cdots + X_{n})\).%
Soit \(\varepsilon \gt 0\).%
%
\begin{enumerate}[label={\arabic*.}]
\item{}Pour \(c\gt0\), on définit \(g: \R \rightarrow \R\) par :%
\begin{equation*}
g(x) = \begin{cases}
x \amp \text{si } |x| \leqslant c, \\
0 \amp \text{sinon.}
\end{cases}
\end{equation*}
Montrer que la variable aléatoire \(g(X_{1})\) est d'espérance finie et que l'on peut choisir \(c\) tel que \(\Es(|g(X_{1}) - X_{1}|) \leqslant \frac{\varepsilon}{2}\).%
\item{}On pose \(a = \Es(g(X_{1}))\). Montrer que :%
\begin{equation*}
\Es(|g(X_{1}) - X_{1} - a|) \leqslant \varepsilon.
\end{equation*}
%
\end{enumerate}
\end{inlineexercise}%
\begin{inlineexercise}{Exercice}{Modèle de Galton-Watson.}{ch-exercices-11}%
On observe des virus qui se reproduisent tous selon la même loi avant de mourir : un virus donne naissance en une journée à \(X\) virus, où \(X\) est une variable aléatoire à valeurs dans \(\N\). Pour tout \(k \in \N\), on note \(\Pr(X = k) = p_{k}\). On suppose \(p_{1}\gt0\) et \(p_{0} + p_{1} \lt 1\). On note \(f\) la fonction génératrice de \(X\). On part au jour zéro de \(X_{0} = 1\) virus. Au premier jour, on a donc \(X_{1}\) virus, où \(X_{1}\) suit la loi de \(X\) ; chacun de ces \(X_{1}\) virus évolue alors indépendamment des autres virus et se reproduit selon la même loi avant de mourir : cela conduit à avoir \(X_{2}\) virus au deuxième jour ; et le processus continue de la sorte. On note \(u_{n} = \Pr(X_{n} = 0)\).%
Calculer \(u_{0}\), \(u_{1}\).%
Montrer que la suite \((u_{n})_{n \in \N}\) est convergente.%
Montrer que pour tout entier \(n \geqslant 0\), on a \(u_{n+1} = f(u_{n})\).%
Que peut-on dire de la limite de \((u_{n})_{n \in \N}\) ? Discuter selon la valeur de \(\Es(X)\). Interpréter le résultat.%
\end{inlineexercise}%
\begin{inlineexercise}{Exercice}{Somme aléatoire de variables aléatoires.}{ch-exercices-12}%
Soit \((X_{n})_{n \geqslant 1}\) une suite de variables aléatoires réelles discrètes, toutes de même loi, et \(N\) une variable aléatoire à valeurs dans \(\N\). On suppose que \(N\) et les variables \(X_{n}\) pour \(n \in \N^{*}\) forment une suite de variables aléatoires indépendantes. On pose :%
\begin{equation*}
\forall n \in \N^{*} \quad S_{n} = \sum_{k=1}^{n} X_{k} \quad \text{et} \quad S_{0} = 0.
\end{equation*}
%
Montrer que \(S_{N}\) est une variable aléatoire.%
Déterminer la loi de \(S_{N}\) lorsque les \(X_{k}\) suivent la loi de Bernoulli de paramètre \(p\) et \(N\) la loi de Poisson de paramètre \(\lambda\).%
Déterminer la loi de \(S_{N}\) lorsque les \(X_{k}\) suivent la loi géométrique de paramètre \(p\) et \(N\) la loi géométrique de paramètre \(p^{\prime}\).%
On suppose que les variables aléatoires \(X_{n}\) sont à valeurs dans \(\N\).%
%
\begin{enumerate}[label={\arabic*.}]
\item{}Montrer que \(G_{S_{N}} = G_{N} \circ G_{X_{1}}\) sur \([0, 1]\).%
\item{}Montrer que, si \(X_{1}\) et \(N\) sont d'espérance finie, alors \(S_{N}\) est d'espérance finie et vérifie la première formule de Wald :%
\begin{equation*}
\Es(S_{N}) = \Es(X_{1}) \Es(N).
\end{equation*}
%
\item{}Montrer que, si \(X_{1}\) et \(N\) possèdent un moment d'ordre 2, alors \(S_{N}\) possède aussi un moment d'ordre 2 et vérifie la seconde formule de Wald :%
\begin{equation*}
\Va(S_{N}) = \Va(X_{1}) \Es(N) + (\Es(X_{1}))^{2} \Va(N).
\end{equation*}
%
\end{enumerate}
\end{inlineexercise}%
\begin{inlineexercise}{Exercice}{Succès consécutifs.}{ch-exercices-13}%
On considère une suite d'épreuves de Bernoulli indépendantes. À chaque épreuve, la probabilité de succès est \(p \in ]0, 1[\). On se donne un entier \(r\) strictement positif. Pour \(n \in \N^{*}\), on note \(\Pi_{n}\) la probabilité qu'au cours des \(n\) premières épreuves, on ait obtenu \(r\) succès consécutifs (au moins une fois).%
Calculer \(\Pi_{0}\), \(\Pi_{1}\), ..., \(\Pi_{r}\).%
Montrer que, pour \(n \geqslant r\), on a \(\Pi_{n+1} = \Pi_{n} + (1 - \Pi_{n-r}) p^{r} (1 - p)\).%
Montrer que la suite \((\Pi_{n})_{n \in \N}\) est convergente. Calculer sa limite.%
Déduire de la question 1 que l'on peut définir une variable aléatoire \(T\) égale au temps d'attente de \(r\) succès consécutifs. On définira \((T = k)\) comme l'événement « on a obtenu des succès aux épreuves de rang \(k - r + 1\), \(k - r + 2\), ..., \(k\) sans jamais avoir obtenu \(r\) succès consécutifs auparavant ».%
Montrer que :%
\begin{equation*}
\Es(T) = \frac{1 - p^{r}}{(1 - p) p^{r}}.
\end{equation*}
%
\end{inlineexercise}%
\begin{inlineexercise}{Exercice}{Marche aléatoire dans \(\mathbb{Z}\) : premier retour à l'origine.}{ch-exercices-14}%
Soit \((X_{n})_{n \in \N^{*}}\) une suite de variables aléatoires, sur le même espace probabilisé \((\Omega, \mathcal{A}, \Pr)\), indépendantes et de même loi définie par :%
\begin{equation*}
\Pr(X_{n} = 1) = p \quad \text{et} \quad \Pr(X_{n} = -1) = 1 - p,
\end{equation*}
où \(p \in [0, 1]\). On pose \(S_{0} = 0\) et, pour tout \(n \in \N^{*}\), \(S_{n} = \sum_{k=1}^{n} X_{k}\). La suite \((S_{n})\) est appelée marche aléatoire dans \(\mathbb{Z}\).%
Déterminer \(u_{n} = \Pr(S_{n} = 0)\) pour tout \(n \in \N\).%
On note \(f(x)\) la somme de la série entière \(\sum u_{n} x^{n}\). Montrer que :%
\begin{equation*}
\forall x \in ]-1, 1[ \quad f(x) = \frac{1}{\sqrt{1 - 4 p (1 - p) x^{2}}}.
\end{equation*}
%
Pour tout entier naturel non nul \(k\), on note \(A_{k}\) l'événement « le mobile retourne pour la première fois à l'origine au bout de \(k\) déplacements », c'est-à-dire :%
\begin{equation*}
A_{k} = (S_{k} = 0) \cap \left(\bigcap_{i=1}^{k-1} (S_{i} \neq 0)\right).
\end{equation*}
On pose \(v_{k} = \Pr(A_{k})\) pour tout \(k \geqslant 1\) et \(v_{0} = 0\).%
%
\begin{enumerate}[label={\arabic*.}]
\item{}Montrer que, pour tout entier naturel \(n\) non nul, on a :%
\begin{equation*}
(S_{n} = 0) = \sum_{k=1}^{n} \Pr((S_{n} = 0) \cap A_{k}).
\end{equation*}
%
\item{}En déduire que, pour tout entier naturel non nul \(n\), on a :%
\begin{equation*}
u_{n} = \sum_{k=0}^{n} u_{n - k} v_{k}.
\end{equation*}
%
\end{enumerate}
\end{inlineexercise}%
\begin{inlineexercise}{Exercice}{Inégalité de Kolmogorov.}{ch-exercices-15}%
Soit \(X_{1}, \ldots, X_{n}\) des variables aléatoires réelles discrètes de l'espace probabilisé \((\Omega, \mathcal{A}, \Pr)\), indépendantes, ayant un moment d'ordre 2, centrées, ainsi que \(a \in \R_{+}^{*}\). On pose, pour tout \(i \in \llbracket 1, n \rrbracket\) :%
\begin{equation*}
S_{i} = X_{1} + \cdots + X_{i}, \quad B_{i} = \left\{\left|S_{1}\right| \lt a\right\} \cap \ldots \cap \left\{\left|S_{i-1}\right| \lt a\right\} \cap \left\{\left|S_{i}\right| \geqslant a\right\}.
\end{equation*}
%
Montrer que, pour \(i \in \llbracket 1, n \rrbracket\), les variables \(S_{i} \mathbf{1}_{B_{i}}\) et \(S_{n} - S_{i}\) sont indépendantes. En déduire que :%
\begin{equation*}
\Es(S_{n}^{2} \mathbf{1}_{B_{i}}) = \Es(S_{i}^{2} \mathbf{1}_{B_{i}}) + \Es((S_{n} - S_{i})^{2} \mathbf{1}_{B_{i}}) \geqslant a^{2} \Pr(B_{i}).
\end{equation*}
%
On pose \(C = \left\{\sup \left(\left|S_{1}\right|, \left|S_{2}\right|, \ldots, \left|S_{n}\right|\right) \geqslant a\right\}\). Montrer que \(\Pr(C) = \sum_{i=1}^{n} \Pr(B_{i})\).%
En déduire l'inégalité de Kolmogorov :%
\begin{equation*}
\Pr\left(\sup \left(\left|S_{1}\right|, \left|S_{2}\right|, \ldots, \left|S_{n}\right|\right) \geqslant a\right) \leqslant \frac{\Va(S_{n})}{a^{2}}.
\end{equation*}
%
\end{inlineexercise}%
\begin{inlineexercise}{Exercice}{Inégalité de Le Cam.}{ch-exercices-16}%
L'objet de l'exercice est d'étudier l'approximation de la loi binomiale par la loi de Poisson. Toutes les variables aléatoires considérées sont définies sur le même espace probabilisé \((\Omega, \mathcal{A}, \Pr)\) et sont à valeurs dans \(\N\).%
Soit \(X\) et \(Y\) deux telles variables aléatoires. Pour tout \(k \in \N\), on pose \(p_{k} = \Pr(X = k)\) et \(q_{k} = \Pr(Y = k)\). On définit la distance entre \(X\) et \(Y\) par :%
\begin{equation*}
\mathrm{d}(X, Y) = \frac{1}{2} \sum_{k=0}^{+\infty} \left|p_{k} - q_{k}\right|.
\end{equation*}
%
%
\begin{enumerate}[label={\arabic*.}]
\item{}Montrer que pour toute partie \(A\) de \(\N\), on a :%
\begin{equation*}
|\Pr(X \in A) - \Pr(Y \in A)| \leqslant \mathrm{d}(X, Y).
\end{equation*}
%
\item{}Démontrer la formule :%
\begin{equation*}
\mathrm{d}(X, Y) = 1 - \sum_{k=0}^{+\infty} \min(p_{k}, q_{k}).
\end{equation*}
%
\item{}En déduire :%
\begin{equation*}
\mathrm{d}(X, Y) \leqslant \Pr(X \neq Y).
\end{equation*}
%
\end{enumerate}
\end{inlineexercise}%
\begin{inlineexercise}{Exercice}{Convergence presque sûre.}{ch-exercices-17}%
Soit \((X_{n})_{n \in \N}\) une suite de variables aléatoires réelles et \(X\) une variable aléatoire réelle définies sur \((\Omega, \mathcal{A}, \Pr)\). On pose :%
\begin{equation*}
B = \left\{\omega \in \Omega \mid \lim_{n \rightarrow +\infty} X_{n}(\omega) = X(\omega)\right\}.
\end{equation*}
Pour tout \(k \in \N^{*}\), on pose :%
\begin{equation*}
C_{k} = \bigcup_{n \in \N} \bigcap_{p \geqslant n} \left(\left|X_{p} - X\right| \leqslant \frac{1}{k}\right).
\end{equation*}
On dit que la suite \((X_{n})_{n \in \N}\) converge presque sûrement vers \(X\) si \(\Pr(B) = 1\).%
Montrer que l'on a \(\Pr(B) = \lim_{k \rightarrow +\infty} \Pr(C_{k})\).%
On suppose que :%
\begin{equation*}
\forall \varepsilon\gt0 \quad \Pr\left(\bigcap_{n \in \N} \bigcup_{p \geqslant n} \left(|X_{p} - X|\gt\varepsilon\right)\right) = 0.
\end{equation*}
Montrer que la suite \((X_{n})_{n \in \N}\) converge presque sûrement vers \(X\).%
Montrer que si la série de terme général \(\Pr(|X_{n} - X|\gt\varepsilon)\) converge pour tout \(\varepsilon\gt0\), alors la suite \((X_{n})_{n \in \N}\) converge presque sûrement vers \(X\).%
\end{inlineexercise}%
\begin{inlineexercise}{Exercice}{Fonction génératrice des moments.}{ch-exercices-18}%
Soit \(X\) une variable aléatoire discrète, pas presque sûrement constante, sur l'espace probabilisé \((\Omega, \mathcal{A}, \Pr)\). On pose, pour \(t \in \R\), \(L_{X}(t) = \Es(e^{t X})\) (la fonction \(L_{X}\) est appelée fonction génératrice des moments de la variable aléatoire \(X\)). On suppose qu'il existe un intervalle \(]\alpha, \beta[\) contenant 0 tel que \(L_{X}(t) \lt +\infty\) pour tout \(t \in ]\alpha, \beta[\).%
Soit \(a \lt b\) deux réels tels que \([a, b] \subset ]\alpha, \beta[\). On considère \(\delta\gt0\) tel que \([a - \delta, b + \delta] \subset ]\alpha, \beta[\). Soit \(k \in \N\). Montrer qu'il existe \(C\gt0\) tel que :%
\begin{equation*}
\forall t \in [a, b] \quad \forall u \in \R \quad |u|^{k} e^{t u} \leqslant C \left(e^{(a - \delta) u} + e^{(b + \delta) u}\right).
\end{equation*}
En déduire que \(X^{k} e^{t X}\) est d'espérance finie pour tout \(t \in ]\alpha, \beta[\).%
Montrer que \(L_{X}\) est de classe \(\mathcal{C}^{\infty}\) sur \(]\alpha, \beta[\) et vérifie :%
\begin{equation*}
\forall t \in ]\alpha, \beta[ \quad \forall k \in \N \quad L_{X}^{(k)}(t) = \Es(X^{k} e^{t X}).
\end{equation*}
En déduire, pour tout \(k \in \N\), une expression du moment d'ordre \(k\) de \(X\). On note \(m\) l'espérance de \(X\).%
\end{inlineexercise}%
\begin{inlineexercise}{Exercice}{Théorème de Weierstrass.}{ch-exercices-19}%
Soit \(f\) une fonction continue de \([0, 1]\) dans \(\R\). Soit \(x \in [0, 1]\). On considère une suite \((X_{n})_{n \geqslant 1}\) de variables de Bernoulli de paramètre \(x\), indépendantes, sur le même espace probabilisé. Pour \(n \geqslant 1\), on pose \(Y_{n} = \frac1n(X_{1} + \cdots + X_{n})\).%
Montrer que :%
\begin{equation*}
\forall(t, u) \in [0, 1]^{2} \quad |f(t) - f(u)| \leqslant \frac{2 \Vert f \Vert _{\infty} (t - u)^{2}}{\eta^{2}} + \varepsilon.
\end{equation*}
%
Montrer que :%
\begin{equation*}
\left|\Es(f(Y_{n})) - f(x)\right| \leqslant \frac{2 \Vert f \Vert _{\infty} \Va(Y_{n})}{\eta^{2}} + \varepsilon \leqslant \frac{2 \Vert f \Vert _{\infty}}{n \eta^{2}} + \varepsilon.
\end{equation*}
%
En déduire que la suite de fonctions polynomiales \((B_n(f))_n\) définie par%
\begin{equation*}
\forall t\in[0,1],\;
B_n(f)(t)\sum_{k=0}^nf(k/n)\binom{n}{k}t^k(1-t)^{n-k}
\end{equation*}
converge uniformément vers \(f\) sur \([0,1]\).%
\end{inlineexercise}%
\begin{inlineexercise}{Exercice}{File d'attente.}{ch-exercices-20}%
Soit \(n\) un entier supérieur ou égal à 2. On considère une file d'attente avec un guichet et \(n\) clients qui attendent. Chaque minute, un guichet se libère. Le guichetier choisit alors le client qu'il appelle selon le processus aléatoire suivant :%
\begin{itemize}[label=\textbullet]
\item{}avec probabilité 2, il appelle le client en première position dans la file,%
\item{}sinon, il choisit de manière équiprobable parmi les \(n - 1\) autres clients.%
\end{itemize}
Enfin, un nouveau client arrive dans la file et se place en dernière position (de telle sorte qu'il y a toujours exactement \(n\) clients qui attendent). Pour tout \(k \in \llbracket 1, n \rrbracket\), on note \(T_{k}\) le temps d'attente d'un client qui se trouve en position \(k\) dans la file.%
Quelle est la loi de \(T_{1}\) ? Donner son espérance, sa variance.%
Montrer que, pour tout \(k \in \llbracket 1, n \rrbracket\), la variable \(T_{k}\) est d'espérance finie.%
Écrire une relation entre \(\Es(T_{k})\) et \(\Es(T_{k-1})\) pour tout \(k \geqslant 2\). En déduire une expression de \(\Es(T_{k})\) en fonction de \(k\) et \(n\). On pourra considérer la suite \(((n + k - 2) \Es(T_{k}))_{1 \leqslant k \leqslant n}\).%
Comparer les caractéristiques de cette file d'attente et d'une file d'attente « classique » (premier arrivé, premier servi).%
\end{inlineexercise}%
\end{chapterptx}
\end{document}
