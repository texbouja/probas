%%%  This is a set of styles to include in the preamble of the generated latex code for the book Discrete Mathematics: an Open Introduction.


\usepackage{bold-extra}
\usepackage{marvosym} %for stop signs.
\usepackage{textcomp}
\usepackage{multicol}



\tcbset{ divisionsolutionstyle/.style={bwminimalstyle, runintitlestyle, exercisespacingstyle, after skip= 1em, title={\space}, breakable, before upper app={\space} } }

\tcbset{ divisionexercisestyle/.style={bwminimalstyle, runintitlestyle, exercisespacingstyle, left=5ex, breakable, before skip=1.5em, before upper app={\setparstyle} } }
\renewtcolorbox{divisionexercise}[4]{divisionexercisestyle, before title={\hspace*{-5ex}\makebox[5ex][l]{#1.}}, title={\notblank{#2}{#2}{}}, after title={}, phantom={\hypertarget{#4}{}}, after={\notblank{#3}{\newline\rule{\workspacestrutwidth}{#3}\newline\vfill}{\par}}}


% % FONT OPTIONS (pick one group to uncomment):

%newpx is my current favorite.  This should work on a TEXLive distribution, but on MiKTeX it initially gave me problems.  Running the update wizard to update MiKTeX then showed ``newpx'' as an available package (only since 8/11/15).  Perhaps just synchronizing in the package manager would do the same thing.  Even then, needed to run initexmf --mkmaps from the command line.
%You could always uncomment all of these to use the default computer modern font.
\usepackage[utf8]{inputenc}
\usepackage[T1]{fontenc}
\usepackage{newpxtext}
\usepackage[vvarbb,cmintegrals,cmbraces,bigdelims]{newpxmath}
\usepackage[scr=rsfso]{mathalfa}% \mathscr is fancier than \mathcal
% \linespread{1.04}         % adds more leading (space between lines)
% quantifiers look strange, so change those back to normal:
	\DeclareSymbolFont{mysymbols}{OMS}{cmsy}{b}{n} %note we make the figures bold to better match newpx.  Replace the ``b'' with an ``m'' to undo this.
	%\SetSymbolFont{mysymbols}  {bold}{OMS}{cmsy}{b}{n}
	%\DeclareSymbolFont{myoperators}   {OT1}{cmr} {m}{n}
	%\SetSymbolFont{myoperators}{bold}{OT1}{cmr} {bx}{n}
	\DeclareMathSymbol{\forall}{\mathord}{mysymbols}{"38}
	\DeclareMathSymbol{\exists}{\mathord}{mysymbols}{"39}
	%\DeclareMathSymbol{\pm}{\mathbin}{mysymbols}{"06}
	%\DeclareMathSymbol{+}{\mathbin}{myoperators}{"2B}
	%\DeclareMathSymbol{-}{\mathbin}{mysymbols}{"00}
	%\DeclareMathSymbol{=}{\mathrel}{myoperators}{"3D}

	% Attempt to add missing unicode character:
	% \usepackage{newunicodechar}
	% \newunicodechar{○}{\circ}

%\usepackage[T1]{fontenc}
%\usepackage{newtxtext,newtxmath}


%\usepackage[bitstream-charter]{mathdesign}
%\usepackage[T1]{fontenc}

%\usepackage[proportional,space,scaled=1.064]{erewhon}
%\usepackage[erewhon,vvarbb,bigdelims]{newtxmath}
%\usepackage[T1]{fontenc}
%\renewcommand*\oldstylenums[1]{\textosf{#1}}



% 
% 
% % % % % % % Other packages % % % % % % % %
% 
% \usepackage{docmute}
% \usepackage{pdfpages}
% 
% \usepackage{svg}
% 
% \usepackage[framemethod=tikz]{mdframed}
% 
% 
% 
% % % % % % % % % % % % % % %  END OF PACKAGES  % % % % % % % % % % % % % % % % % % %
% 
% %%%%%%%%%%%%%%%%%%%%%%%%%%%%%%%%%%%%%%%%%%%%%%%%%%%%%%%%%%%%%%%%%%%%%%%%
% %%%%%%%%%%%%%%%%%  Nicely styled environments: %%%%%%%%%%%%%%%%%%%%%%%%%
% %%%%%%%%%%%%%%%%%%%%%%%%%%%%%%%%%%%%%%%%%%%%%%%%%%%%%%%%%%%%%%%%%%%%%%%%
% 
% Tweak exercises styling
% \renewtcolorbox{divisionexercise}[4]{bwminimalstyle, runintitlestyle, exercisespacingstyle, left=4ex, breakable, parbox=false, before title={\hspace{-4ex}\makebox[4ex][l]{#1.}}, title={\notblank{#2}{#2\space}{}}, phantom={\hypertarget{#4}{}},  after={\notblank{#3}{\newline\rule{\workspacestrutwidth}{#3\textheight}\newline}{\vskip 1em}}}

% \setlist{itemsep=2pt plus 1pt minus 1pt}

% \reversemarginpar


% These are now all part of pretext-latex-dmoi or custom-latex.
% 
% %create a mdframe style for examples:
% \mdfdefinestyle{examplestyle}{%
% 	leftmargin=1.5ex, %left/right margins for oneside
% 	rightmargin=1.5ex,
% 	outermargin=1.5ex, %inner/outer margins for twoside
% 	innermargin=1.5ex,
% 	innertopmargin=0pt,
% 	innerbottommargin=1.5ex,
% 	skipbelow=1.5em,
% 	roundcorner=0pt,
% 	topline=false,
% 	rightline=false,
%   frametitleaboveskip=2pt,
% 	frametitlebelowskip=2pt,
% 	needspace=3em,
% }
% 
% %Create indented ``example'' environment:
% \mdtheorem[style=examplestyle]{mdexample}[cthm]{Example}
% 
% %Redefine ``example'' to call the new mdexample, and put a paragraph break after it.
% \renewenvironment{example}{%
% \mdexample
% }{%
% \endmdexample\par
% }
% 
% 
% % \LetLtxMacro\mdexample\oldexample
% % \let\endmdexample\endoldexample
% %
% % \newenvironment{mdexample}{%
% % \oldexample
% % }{%
% % \endoldexample\par
% % }
% 
% %Create ``investigation'' environment for in-text worksheet type problems:
% \renewenvironment{investigation}{%
% \mdfsetup{%
% 	frametitle={\colorbox{white}{\large\space\space \textit{Investigate!} \space} },
% 	frametitlealignment=\centering,
% 	frametitleaboveskip=-1.5ex,
% 	frametitlebelowskip=0pt,
% 	frametitlealignment={\hspace*{2pt}},
% 	innerbottommargin=2em,
% 	innermargin=1ex,
% 	outermargin=1ex,
% 	leftmargin=1ex,
% 	rightmargin=1ex,
% %	topline=false,
% %	bottomline=false,
% 	linecolor=green!50!black,
% 	linewidth=1pt,
% 	skipabove=2em,
% 	skipbelow=2em,
% 	roundcorner=15pt,
% 	needspace=2em,
% }
% \vskip 1em
% \begin{mdframed}
% }{%
% \end{mdframed}\vspace{-4em} \begin{center}\colorbox{white}{ \space\space\quad {\LARGE \color{red} \Stopsign} \quad \textbf{Attempt the above activity before proceeding} \quad {\LARGE\color{red} \Stopsign}\quad\space\space }  \end{center}\par
% }
% 
% 
% %% begin: assemblage
% %% minimally structured content, high visibility presentation
% %% One optional argument (title) with default value blank
% %% 3mm space below dropped title is increase over 2mm default
% \renewtcolorbox{assemblage}[1][]
%   {breakable, skin=enhanced, boxrule=0.5pt, sharp corners, colback=blue!5, colframe=blue!15,
%    colbacktitle=blue!20, coltitle=black, boxed title style={sharp corners, frame hidden},
%    fonttitle=\bfseries, attach boxed title to top left={xshift=4mm,yshift=-3mm}, top=3mm, title={#1}}
% 
% 
% 
% 	 %% Divisional exercises are rendered as faux list items
% 	 %% with hard-coded numbers as arguments, not as LaTeX environments
% 	 \RenewDocumentEnvironment{divisionexercise}{mo}
% 	  {\begin{itemize}\item[\textbf{#1}.]\IfValueT{#2}{\textbf{#2}~}}
% 	  {\end{itemize}}
% %%%%%%%%%%%%%%%%%  End environments %%%%%%%%%%%%%%%%%%%%%%%%%%
% 
% Start sections on a new page.  Add \clearpage to the second command to start subsections on their own page.
\newcommand{\sectionbreak}{\ifnum\value{section}>1\clearpage\thispagestyle{plain}\fi\phantomsection}
\newcommand{\subsectionbreak}{\phantomsection}
% 
% % use QED to end proofs:
% \usepackage{amsthm}
% \renewcommand{\qedsymbol}{\textsc{qed}}
% 
% 
% %%%%%%%% Set description list spacing the way I want %%%%%%%%%%%%%%%
% \setlist[description]{style=nextline, leftmargin=3em}
% 
% 
% %%%%%%% Set chapters to start at 0 %%%%%%%%%%%%%%%%%%%%%%%%%%%
% \setcounter{chapter}{-1}
% 
% 
% %Fix widows and orphans (single lines at top/bottom of page):
% \clubpenalty=10000
% \widowpenalty=10000
% \raggedbottom
% 
% 
% 
% %%%%%%%%%%%%%%%%%%%%%%%%%%%%%%%%%%%%%%%%%%
% %%%%%%%  Headers and footers %%%%%%%%%%%%%
% %%%%%%%%%%%%%%%%%%%%%%%%%%%%%%%%%%%%%%%%%%
% 
% \usepackage{fancyhdr}
% \pagestyle{fancy}
% \renewcommand{\chaptermark}[1]{\markboth{\thechapter.\ #1}{}} %Removes word "chapter" from the \leftmark.

% \fancyhead{} % clear header fields
% \fancyhead[LE]{{\footnotesize \textsl{\thepage}}~~ \textsc{\scriptsize \nouppercase{\leftmark}}}
% \fancyhead[RO]{\textsc{\scriptsize \nouppercase{\rightmark}} ~~ {\footnotesize \textsl{\thepage}}  }
% \fancyfoot{}
% 
% 
% 
% 
% 
% %%%%%%%%%%%%%%%%%%%%%%%%%%%%%%%%%%%%%%%%%%
% %%%%%%%    Chapter headings  %%%%%%%%%%%%%
% %%%%%%%%%%%%%%%%%%%%%%%%%%%%%%%%%%%%%%%%%%
% 
% % \usepackage[bf,sc,center,outermarks]{titlesec}
% 
% 
% %%%%% FIX FOR BUG IN TITLESEC %%%%%%%%%%%
% \usepackage{etoolbox}
% 
% \makeatletter
% \patchcmd{\ttlh@hang}{\parindent\z@}{\parindent\z@\leavevmode}{}{}
% \patchcmd{\ttlh@hang}{\noindent}{}{}{}
% \makeatother
% %%%%% END FIX %%%%%%%%%%%%%%%%%%%%%%%%%%%%
% 
% 
% \titleformat{\chapter}[display]
% 	{\Large\filcenter}
% 	{\rule[4pt]{.3\textwidth}{2pt} \hspace{2ex} \large\textsc{\chaptertitlename} \thechapter \hspace{3ex} \rule[4pt]{0.3\textwidth}{2pt} }
% 	{1pc}
% 	{\titlerule\vspace{1ex}\huge\textsc}
% 	[\vspace{.75ex}\titlerule]
% \titlespacing*{\chapter}{0pt}{-2em}{2em}
% 
% \titleformat{\paragraph}[block]
%   {\normalfont\bfseries\filcenter}
%   {\theparagraph}
%   {}
%   {\textsc}
% 

% %%%%%%%%%  End chapter/sectio headings %%%%%%%%%%%%%%%%%
% 

%%% End of File.
